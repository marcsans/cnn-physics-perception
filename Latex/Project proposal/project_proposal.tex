\documentclass[11pt, oneside]{amsart}
\usepackage{geometry}
\geometry{letterpaper}
\usepackage[francais]{babel}
\usepackage[utf8]{inputenc}
\usepackage{graphicx}
\usepackage{float}
\usepackage{etex,mathtools}
\usepackage{amssymb}
\usepackage{enumitem}
\usepackage{amsmath}
\usepackage{caption}
\usepackage{listings}
\usepackage{array}
\usepackage{hyperref}

%opening
\title{Project Proposal : Perceiving Physics by Integrating a Physics Engine with Deep Learning}
\author{Louis Thiry, Marc Sanselme}

\begin{document}

\maketitle

We intend to follow the points that are in the project proposal "Perceiving physics".

\begin{enumerate}
  \item \textbf{Physical system}.

    We will work with a double pendulum with bouncy walls as proposed in the point 5.
  \item \textbf{Simuator}.

    We will write a simple double pendulum motion simulator in python.
  \item \textbf{Video}.

    We will record a video of such a double pendulum with bouncy wall
  \item \textbf{Hypercolumns}.

    We have begun to work on the Hypercolumns extraction problem.
    We have found a begin of python code that does hypercolumn extracting on \href{http://blog.christianperone.com/2016/01/convolutional-hypercolumns-in-python/}{\textbf{this page}}.
    We had to modify the code in order to get it working on our computers.
    We are currently testing it on a pendulum video that we downloaded to see which layer is responsive to the mass.
    Once we have the correct hypercolumn features, we will compute the mass position.
  \item \textbf{Inference}.

    The reason why we haven't chose the simple pendulum is that we can't infer the mass of the ball.
    Indeed, if we neglet the friction effects, the equations of pendulum motion do not depend on the mass.
    The rebound equation do not depend on the mass either.
    So we can't infer the mass of the pendulum.
    We can infer :
    \begin{itemize}
      \item the length, but we can also deduce it from the frames directly.
      \item the gravity, but this value does not depend on the pendulum.
      \item the resitution coefficient.
    \end{itemize}
    Thus we have chosen to work on a double pendulum.
    On that system, we will try to infer the mass of one of the balls.
\end{enumerate}

\end{document}
